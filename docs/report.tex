
% ****** Start of file aipsamp.tex ******
%
%   This file is part of the AIP files in the AIP distribution for REVTeX 4.
%   Version 4.1 of REVTeX, October 2009
%
%   Copyright (c) 2009 American Institute of Physics.
%
%   See the AIP README file for restrictions and more information.
%
% TeX'ing this file requires that you have AMS-LaTeX 2.0 installed
% as well as the rest of the prerequisites for REVTeX 4.1
%
% It also requires running BibTeX. The commands are as follows:
%
%  1)  latex  aipsamp
%  2)  bibtex aipsamp
%  3)  latex  aipsamp
%  4)  latex  aipsamp
%
% Use this file as a source of example code for your aip document.
% Use the file aiptemplate.tex as a template for your document.
\documentclass[%
notitlepage,
 %reprint,%
%author-year,%
%author-numerical,%
%draft,
]{revtex4-1}

\usepackage{graphicx}% Include figure files
\usepackage{dcolumn}% Align table columns on decimal point
\usepackage{bm}% bold math

% packages added by Anand
\usepackage{subfig}
\usepackage{color}
\usepackage{amsmath}
\usepackage{amssymb}
\usepackage{xspace}
\usepackage{multirow}
\usepackage{bm}
%\usepackage[mathlines]{lineno}% Enable numbering of text and display math
%\linenumbers\relax % Commence numbering lines
\usepackage[]{breqn}

\everymath{\displaystyle}

\newcommand{\pd}[2]{\frac{\partial #1}{\partial #2}}
\newcommand{\be}{\ensuremath{{\beta}}\xspace}
\newcommand{\ret}{\ensuremath{ Re_{\theta} \xspace}}
\renewcommand{\bm}{\ensuremath{{\boldsymbol\beta_{MAP}}}\xspace}
\newcommand{\mat}[1]{{\ensuremath{\bf{ #1}}}}
% comment the following lines to see figures
%\usepackage{ifdraft}
%\ifdraft{\renewcommand{\includegraphics}{\relax}}{\relax}
%\renewcommand{\includegraphics}[2][]{}
%\renewcommand{\subfloat}[0]{}
\usepackage{url}
\begin{document}

\preprint{}

\title[]{One dimensional Premixed flame }% Force line breaks with \\
%\thanks{Footnote to title of article.}

\author{Anand Pratap Singh}
\email{anandps@umich.edu}
\affiliation{Department of Aerospace Engineering, University of Michigan, Ann Arbor, MI 48109, USA}%

\date{\today}% It is always \today, today,
             %  but any date may be explicitly specified

\begin{abstract}

\end{abstract}

%\pacs{Valid PACS appear here}% PACS, the Physics and Astronomy
                             % Classification Scheme.
%\keywords{Suggested keywords}%Use showkeys class option if keyword
                              %display desire\maketitle
%\tableofcontents
\vspace{1cm}
\texttt{code: https://github.com/anandpratap/flame\_1d}
\section{Introduction}
The governing equations are,
\begin{eqnarray}
  \dot{m}\pd{T}{x} = \pd{}{x}\left(\alpha\pd{T}{x}\right) + \dot{\omega}_T(\mat{Y}, T)\\
  \dot{m}\pd{Y_{k}}{x} = \pd{}{x}\left(D\pd{Y_{k}}{x}\right) + \ddot{\omega}_{Y_k}(\mat{Y}, T)\\
\end{eqnarray}
Density is assumed to be constant (=1). The diffusivisty are assumed to be constant, $\alpha = D = 1.5\times 10^{-5}$. Auxillary relations for the calculation of the source terms are,
\begin{eqnarray}
&&  \ddot{\omega}_{Y_k}(\mat{Y}, T) = \sum_{reactions}W_{k}\dot{\omega}_k\\ 
&&  \dot{\omega}_T(\mat{Y}, T) = \sum_{reactions}\frac{1}{C_p}\sum_{k}\ddot{\omega}_{Y_k}h_k\\
&&  \dot{\omega}_T(\mat{Y}, T) = \frac{Q q}{C_p} \text{(for the base case used for demo, as the above expression will lead to zero source)}\\
&&  q = k_f\prod_k [X_k]^{\nu_{lhs,k}} - k_b\prod_k [X_k]^{\nu_{rhs,k}}\\
&&  X_k = \frac{Y_k/W_k}{\sum_j Y_j/W_j}\\
&&  \dot{\omega}_{k} = \left(\nu_{rhs,k} - \nu_{lhs,k}\right)q
\end{eqnarray}
The initial conditions should be carefully choosen as the solution depends on them.


\end{document}

%
% ****** End of file aipsamp.tex ******
